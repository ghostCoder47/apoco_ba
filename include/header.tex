

%koma script config
\KOMAoptions{cleardoublepage=empty}

\KOMAoptions{%	
	toc=indented,
	toc=bib,
	bibliography=totoc, % add Bibliography to TOC without number
	bibliography=openstyle,
	toc=listof,
	listof=totoc, % add Lists to TOC without number
	listof=left, % tabular styles
	listof=indented, % hierarchical style
	listof=chaptergapsmall % New chapter starts are marked by a gap of a smallsingle line
	%numbers=noenddot %dot behind section numbers (enddot, noenddot, autoenddot)
}





%

% Abstaende zwischen Caption und Bild/Tabelle

%

\setlength\abovecaptionskip{0.4em}

\setlength\belowcaptionskip{0.2em}

\raggedbottom

\setlength{\parskip}{2.0ex}

\setlength{\parindent}{0.0cm}

\parindent 0pt

\setcounter{tocdepth}{2}

% kopf fuss - zeile
\pagestyle{fancy}

\rhead[\leftmark]{\thepage}
\lhead[\thepage]{\rightmark}
\chead{}
\rfoot[]{\thepage}
\lfoot[\thepage]{}
\cfoot{}

\renewcommand{\chaptermark}[1]{\markboth{#1}{}}
\renewcommand{\sectionmark}[1]{\markright{\thesection\ #1}}

%author angaben
\newcommand{\bachelorarbeittitelDE}{Entwurf und Implementierung einer Anwendung f\"ur ambulante Betreuung bei Adipositas}
\newcommand{\bachelorarbeittitelENG}{Design and Implementation of an Ambulant Care Application of Obesity}
\newcommand{\namedesauthors}{Dawid Janas}
\newcommand{\geburtsdatum}{22.08.1978}
\newcommand{\geburtsort}{Luban / Polen}
\newcommand{\arbeitart}{Bachelorarbeit}

%hyperref optionen
\hypersetup{
	pdftitle={Bachelorarbeit - \namedesauthors},
	pdfauthor={\namedesauthors}
}

%path
%\graphicspath{{screenshots/}{diagramme/}{fotos/}}


%
% define colors
%
\definecolor{codecolor}{rgb}{0.85,0.85,0.85}
\definecolor{commentcolor}{rgb}{0.0,0.6,0.0}

%
% listings package options
%
\lstset{ %
	language=[Visual]C++,			% choose the language of the code
	basicstyle=\scriptsize,		% the size of the fonts that are used for the code
	numbers=left,					% where to put the line-numbers
	numberstyle=\scriptsize,		% the size of the fonts that are used for the line-numbers
	stepnumber=1,					% the step between two line-numbers. If it's 1 each line will be numbered
	numbersep=5pt,					% how far the line-numbers are from the code
	backgroundcolor=\color{white},	% choose the background color. You must add \usepackage{color}
	showspaces=false,				% show spaces adding particular underscores
	showstringspaces=false,			% underline spaces within strings
	showtabs=false,					% show tabs within strings adding particular underscores
	frame=single,					% adds a frame around the code
	tabsize=2,						% sets default tabsize to 2 spaces
	captionpos=b,					% sets the caption-position to bottom
	breaklines=true,				% sets automatic line breaking
	breakatwhitespace=false,		% sets if automatic breaks should only happen at whitespace
	escapeinside={\%*}{*)},         % if you want to add a comment within your code
	morekeywords={SLOT,SIGNAL,var,function}            % if you want to add more keywords to the set
	keywordstyle=\color{blue},
	commentstyle=\color{commentcolor},
	stringstyle=\color{red},
	morecomment=[s][\color{blue}]{\#}{\ }
}

%
%   Hintergrundfarbe von Quellcode
%
\lstloadlanguages{[Visual]C++, Java, PHP}
\lstset{basicstyle = \ttfamily \scriptsize}
\lstset{backgroundcolor=\color{codecolor}}
\lstset{extendedchars=true} \lstset{showstringspaces = false}
\lstset{morekeywords={SLOT,SIGNAL,slot,signal,:,signals,slots,var,function}}

 
