%einleitung.tex

\chapter{Einleitung} 

\section{Motivation}

Adipositas gilt in Deutschland als eine chronische Gesundheitsst\"orung.
Diese geht aus einer falschen Ern\"ahrung und Stoffwechselproblemen hervor.
Dabei leiden die Patienten an starkem \"Ubergewicht.
Mangelnde Bewegung durch sitzende T\"atigkeiten am Arbeitsplatz und eine passive Freizeit beg\"unstigen zus\"atzlich die Entwicklung von \"Ubergewicht.
Bleibt Adipositas unbehandelt, so kann es  zu schwerwiegenden Folgeerkrankungen f\"uhren wie etwa Bluthochdruck,
Verkalkung der Herzkranzgef\"a\ss{}e, Zuckerkrankheit, Krebs, aber auch psychische Leiden nach sich ziehen.
Hilfe findet man in Kliniken, Beratungsstellen, Selbsthilfe- und Wohngruppen.
In der heutigen Zeit besitzt fast jeder B\"urger ein Smartphone.
Android ist aktuell mit 79,3\% Marktanteil\cite{AndroidAnteile:01}  das meist verbreitete Smartphone-Betriebssystem der Welt.
Es bietet aufgrund seiner vielseitigen Konnektivit\"at durch Technologien wie WLAN, 3G/4G und Bluetooth 
eine exzellente Grundlage f\"ur die Entwicklung von Softwaresystemen zur Unterst\"utzung von Gesch\"aftsprozessen.
Einige Hersteller haben sich bereits darauf spezialisiert sogenannte \emph{E-Helth} Systeme anzubieten.
Dabei soll mit Hilfe von elektronischen Ger\"aten die Qualit\"at zur medizinischen Versorgung im Gesundheitswesen gesteigert werden. \\
Das Ziel der Bachelorarbeit ist, die Entwicklung und Realisierung eines mehrschichtigen Softwaressystems
zum Erfassen von Messdaten und Unterst\"utzung von Betreuer und Patient.




\section{Aufgabenstellung im Detail}

\subsection*{Praxisprojekt}

In dem Praxisprojekt soll eine Android-Anwendung zum Monitoring der t\"aglichen Kalorienzufuhr von Patienten entwickelt und implementiert werden.
Dabei wird die Anwendung eine Handykamera als Barcodescanner nutzen, um Lebensmitteleigenschaften aus dem Internet von einer Datenbank abzufragen.
\"Uber die Differenzw\"agung und die Lebensmitteleigenschaften wird die entsprechende Kalorienzahl der entnommenen Menge
in einem Tagesprotokoll festgehalten.
Die Wiegedaten sollen \"uber Bluetooth \"ubertragen werden. 
Die aufgezeichneten Tagesprotokolle werden zwecks sp\"aterer Auswertem\"oglichkeit in eine Webserver Datenbank gesendet.
F\"ur die Auswertung soll eine Webanwendung entwickelt und implementiert werden, welche die Tagesprotokolle \"ubersichtlich veranschaulicht.


\subsection*{Bachelorarbeit}

W\"ahrend der Bachelorarbeit wird die in dem Praxisprojekt erstellte Arbeit erweitert und verbessert.
Die Software soll Adipositaspatienten helfen durch eine \"Ubersichtsfunktion den Di\"atplan einzuhalten.
Dem behandelnden Arzt soll die Software erm\"oglichen das Ern\"ahrungsverhalten seiner Patienten zu \"uberwachen und bei einer ambulanten 
Adipositas-Betreuung zu unterst\"utzen.

Zus\"atzlich zur Ermittlung der t\"aglichen Kalorienzufuhr wird 
nun auch das K\"orpergewicht und der Blutdruck inklusive Puls erfasst und die Tagesprotokolle dadurch erweitert. 
F\"ur das Erfassen der Daten sollen eine Bluetooth-f\"ahige K\"orperwaage und ein Blutdruckmessger\"at genutzt werden. 
Die webseitige Anwendung wird mit der Ajax-Architektur umgesetzt.
Die aufgezeichneten Daten sollen in einem Graphen, der \"uber ein Kalender gesteuert wird, dargestellt werden. 
Der Arzt soll die M\"oglichkeit haben, die Aufl\"osung der dargestellten Daten zwischen Tag, Woche und Monat umzuschalten und 
\"uber ein Kalender-Widget navigieren zu k\"onnen.
Die Navigation erm\"oglicht es zum gew\"unschten Datum zu springen und das Wochenende kann an eine beliebige Stelle in der Darstellung ger\"uckt werden.
Alle Daten werden als Liniendiagramme oder Liste dynamisch erstellt und visualisiert.
Somit wird auch eine detailierte Einsicht in die Zusammensetzung der Mahlzeiten erm\"oglicht.
Als weiterer wichtiger Aspekt soll ein Konzept von Sicherheitsmechanismen f\"ur die Endpunkt zu Endpunkt Verschl\"usselung der Daten ausgearbeitet werden.
Das betrifft die Daten\"ubertragung vom Handy zum Server und vom Server zur Monitoring Software.
 

\section{Aufbau der Arbeit}

\begin{itemize}
 \item Kapitel 1:\\
 Das Kapitel 1 umfasst die Einleitung und gibt einen \"Uberblick \"uber die geleistete Arbeit und Motivation 
 w\"ahrend der Zeit im Praxisprojekt und der Bacherlorarbeit.
 
 \item Kapitel 2:\\
 In diesem Kapitel werden die notwendigen Grundlagen f\"ur die verwendete Hardware, 
 Technologien und Standards erl\"autert,
 die wichtig f\"ur die Durchf\"uhrung des Projekts waren. 
 Es wird unter anderem die Umsetzung von Ajax in einer Webanwendung anhand von Beispielen erkl\"art,
 wie eine REST-Schnittstelle funktioniert oder ein EAN-Code aufgebaut ist.
 
 \item Kapitel 3:\\
 Dieses Kapitel stellt Verschl\"usselungsmechanismen vor,
 mit denen eine sichere Daten\"ubertragung erm\"oglicht werden soll.
 Es werden Hashfunktionen sowie Zwei-Wege-Verschl\"usselungen vorgestellt.
 Au\ss{}erdem wird ein Konzept f\"ur eine sichere Web-API und eine verschl\"usselte Kommunikation einer Android-Anwendung ausgearbeitet.
 
 \item Kapitel 4:\\
 In diesem Kapitel werden Designentscheidungen bei der Modellierung der Android-Anwendung erl\"autert.
 Es werden die verwendeten Architekturen und die Modellierung der persistierenden Schicht verdeutlicht.
 Dar\"uber hinaus werden Hintergrundprozesse der Anwendung anhand von Beispiel-Codes und Diagrammen erkl\"art.
 Am Ende wird auf die graphische Oberfl\"ache n\"aher eingegangen.
 
 \item Kapitel 5:\\
 Das Kapitel 5 besch\"aftigt sich mit der Webanwendung.
 Wie schon bei der Android-Anwendung werden hier Designentscheidungen w\"ahrend der Softwareentwicklung besprochen.
 Es wird auf die Umsetzung der Darstellung von Informationen in einem Graph und damit zusammenh\"angende Steuerprozesse durch ein Kalender-Widget eingegangen.
 
 \item Kapitel 6:\\
 Dieses Kapitel beinhaltet eine Zusammenfassung der erarbeiteten Ergebnisse
 und einen Ausblick auf m\"ogliche Erweiterungen.
\end{itemize}
