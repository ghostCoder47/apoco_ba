

\chapter{Zusammenfassung / Ausblick} 


W\"ahrend der Praxisarbeit und der Bachelorthesis wurde ein vielschichtiges Softwaresystem entwickelt,
welches gleich mehrere bekannte und wichtige Architekturen, wie MVC und Client-Server und Technologien, 
wie zum Beispiel Ajax vereint. 
F\"ur die Erweiterung des Smartphones um Messsensoren, wie die K\"orperwaage oder das 
Blutdruckmessger\"at wurde die Bluetooth-Schnittstelle verwendet.
Die Daten\"ubertragung von Smartphone an einem Webserver geschieht unter dem Einsatz der REST-Architektur
und die Messwerte werden auf einen MySQL-Datenbankserver persistent gespeichert.
Die Benutzerschnittstelle der Webanwendung ist mit Ajax realisiert, was f\"ur mehr Interaktivit\"at und 
eine bessere User Experience sorgt, da die Software aktiv auf den Benutzer reagiert.
Dadurch bekommt der Benutzer das Gef\"uhl die Anwendung w\"urde wie eine Desktop-Anwendung direkt auf 
seinem Rechner ausgef\"uhrt werden.
Dabei muss die Software nicht auf einem Rechner installiert werden, sondern hier reicht schon ein 
Ajax-f\"ahiger Webbrowser.
Somit besteht auch nicht die Gefahr, dass die Software durch Viren gesch\"adigt werden kann.
Die Android-Anwendung benutzt die Kamera auf der R\"uckseite des Smartphones als Barcodescanner.
Damit wird es m\"oglich auf eine bequeme Art und Weise Lebensmittel zu identifizieren, ohne, dass man 
m\"uhsam die EAN-Nummer oder sogar die Lebensmitteleigenschaften manuell eingeben muss. 
Die Daten werden \"uber das Internet von einem Datenbankserver abgefragt
und verwendet. 
Da die Messsensoren in die Software \"uber Bluetooth-Technologie eingebunden sind,
spart sich der Benutzer das Ablesen und Eingeben der Werte mit der Hand.
Die Daten werden sicher \"uber Bluetooth \"ubertragen und in der Anwendung verwaltet.
Die Software ist an Menschen mit einer gesundheitlichen Einschr\"ankung gerichtet
und durch die erw\"ahnten F\"ahigkeiten hat sie eine unterst\"utzende Wirkung f\"ur den Benutzer.
Sein Aufwand wird durch den Einsatz der erw\"ahnten Technologien minimiert, was ihn somit nicht 
zus\"atzlich belastet.
Dadurch, dass die Tagesprotokolle \"uber das Internet an den behandelnden Arzt geschickt werden, 
verringert sich der Stressfaktor f\"ur den Patienten, da er nicht zu jeder Untersuchung einreisen muss.
Auf der anderen Seite wird auch der Aufwand f\"ur den Arzt minimiert, der in besonders schweren F\"allen 
auch \"oferts einen Hausbesuch machen muss.
Durch die \"Ubersichtsfunktion und farbliche Untermalung der Messwerte bei Gefahren, 
wird der Arzt auf die Patienten, die seine Betreuung im besonderen Ma\ss{}e erfordern, aufmerksam gemacht.
Die Software steigert somit auch die Effizienz der Betreuung, da der Arzt sich mehr auf hochgradig 
relevante F\"alle konzentrieren kann.
Die verschiedenen Messsensoren f\"uhren zu einer aussagekr\"aftigeren Diagnose.
Sollte der Patient sich nicht an die Vorgaben des Arztes halten und es vers\"aumen seine Mahlzeiten 
gr\"undlich zu protokollieren, so w\"urden andere Messwerte \"uber sein K\"orpergewicht und seinen Blutdruck 
darauf hinweisen.
Es ist dennoch von gro\ss{}er Bedeutung, dass die Zusammenarbeit zwischen Arzt und Patient stimmig ist.
Der Patient muss diszipliniert mit der Protokollierung umgehen und ehrlich sein,
damit die Messwerte eine korrekte Aussage nicht verwischen.
Mit der ambulanten Betreuung durch den Arzt und der Protokollierung hat der Patient eine bessere 
\"Ubersicht und kann sich selbst besser kontrollieren, da er darauf aufmerksam gemacht wird, 
wenn er zum Beispiel zu viel Nahrung zu sich nehmen sollte.
Die Best\"atigung, dass sein Gesundheitszustand sich nach und nach bessert, wirkt auf den Patienten mit 
Sicherheit auch positiv und es f\"ordert seine Genesung auf psychischer Ebene, 
da er Resultate protokolliert vor sich hat.
Die Arbeit hat gezeigt, dass ein handels\"ubliches Smartphone, ausgestattet mit den \"ublichen 
Standardschnittstellen, in ein sehr umfangreiches und seri\"oses Projekt, 
welches \"uber die Smartphone-Grenzen hinaus reicht, integriert werden kann.
Nach Aussagen des Marktforschungsunternehmens \emph{International Data Corporation} (IDC)\cite{AndroidAnteile:01},
ist Android mit 79,3\% das f\"uhrende Smartphonebetriebssystem weltweit. 
Die Entscheidung, eine Software f\"ur Android zu implementieren, war deshalb richtig.
Auf diese Weise wird eine breite Masse der Smartphonebenutzer erreicht.
W\"ahrend der Bachelorarbeit entstanden einige Entwicklungsprobleme zwischen der Bodytel WeightTel 
K\"orperwaage und der Android-Anwendung.
Im Projekt wurde als Entwicklungsger\"at ein \emph{Samsung Galaxy S3} genutzt.
Das Smartphone wurde schon zu Beginn mit der damals neusten Android-Version 4.1 aktualisiert.
Es stellte sich heraus, dass die K\"orperwaage nicht mit dieser Android-Version zusammenarbeiten kann.
Nachforschungen haben ergeben, dass bis zur Android-Version 4.04, die Firma Google f\"ur die Bluetoothfunktionalit\"at
die freie Software \emph{BlueZ} \cite{BlueZ:01} verwendet hat.
\emph{BlueZ} ist unter Linux die Standardimplementierung der Protkolle f\"ur den Bluetooth Stack.  
Die h\"oheren Android-Versionen verwenden die Implementierung von \emph{Broadcom} \cite{Broadcom:01}.
Der Broadcom Stack soll f\"ur Android speziell optimiert worden sein.
Leider gibt es Ger\"ate, wie die WeightTel K\"orperwaage, die damit nicht kompatibel sind.
Zur L\"osung des Problems im Projekt wurde ein Downgrade auf Android 4.04 vollzogen.
Dies wurde auch vom Hersteller Bodytel empfohlen.
Da die Messger\"ate auch in Zukunft verwendet werden sollten, muss eine andere L\"osung entworfen werden.
Hier ergeben sich mehrere Ans\"atze.
Die einfachste M\"oglichkeit w\"are, dem Benutzer eine manuelle Messdateneingabe anzubieten.
Dieser w\"urde die Messwerte ablesen und sie in der Anwendung eintragen.
Eine weitere M\"oglichkeit ist eine Art Homeserver zu entwickeln.
Dieser Homeserver kann zum Beispiel auf einem Minicomputer, wie dem \emph{Raspberry Pi} \cite{rasppi:01} basieren.
Der \emph{Raspberry Pi} w\"urde mit Linux arbeiten, wobei er als Basis bereits den BlueZ Stack nutzt.
Der Homeserver w\"are eine zentrale Stelle, um die Messger\"ate zu erfassen 
und w\"urde anschlie\ss{}end die Messwerte an das Smartphone \"uber Bluetooth senden.
Ein dritter Vorschlag w\"are, einen eigenen Bluetooth Stack unter Verwendung der BlueZ-Bibliotheken zu entwickeln.
Der Stack w\"urde von der Android-Anwendung benutzt werden. Dabei k\"onnte er nach Au\ss{}en mit allen Messger\"aten
kommunizieren und die Kommunikation an die Android-Anwendung weitergeben.
Intern w\"urde die Anwendung weiterhin mit dem Bluetooth-Modul des Ger\"ates arbeiten.
Beim Ausblick in die Zukunft k\"onnte die Software mit weiterer Funktionalit\"at und Messsensoren ausgebaut werden.
Zum Beispiel durch das Erfassen von Blutzuckerwerten oder als Langzeit-EKG.
Hierbei sollte eine Bedarfsanalyse durchgef\"uhrt und ein Konzept f\"ur eine umfangreiche Plattform zur 
F\"orderung der Gesundheit ausgearbeitet werden.
