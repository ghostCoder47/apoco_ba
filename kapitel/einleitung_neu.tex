%einleitung.tex

\chapter{Einleitung} 

\section{Motivation}

Adipositas gilt in Deutschland als eine chronische Gesundheitsst\"orung.
Diese geht aus einer falschen Ern\"ahrung und Stoffwechselproblemen herfor.
Dabei leiden die Patienten an starkem \"Ubergewicht.
Mangelnde Bewegung durch sitzende T\"atigkeiten am Arbeitsplatz und eine passive Freizeit beg\"unstigen zus\"atzlich die Entwicklung von \"Ubergewicht.
Bleibt Adipositas unbehandelt, so kann es  zu schwerwiegenden Folgeerkrankungen f\"uhren wie etwa Bluthochdruck,
Verkalkung der Herzkranzgef\"a\ss{}e, Zuckerkrankheit, Krebs aber auch psychische Leiden nach sich ziehen.
Hilfe findet man in Kliniken, Beratungsstellen, Selbsthilfe- und Wohngruppen.
In der heutigen Zeit Besitzt fas jeder B\"urger ein Smartphone.
Android ist aktuell mit 79,3\% Marktanteil das meist verbreitete Smartphone-Betriebssystem der Welt.
Es bietet auf Grund seiner vielseitigen Konnektivit\"at durch Technologien wie WLAN, 3G/4G und Bluetooth 
eine exzelente Grundlage f\"ur die Entwicklung von unterst\"utzenden Systemen von Gescheftsprozessen.
Einige Hersteller haben sich bereits darauf spezialisiert, so genannte \emph{E-Helth} Systeme anzubieten.
Dabei soll mit Hilfe von elektronischen Ger\"aten die Qualit\"at zur medizinischen Versorgung und anderen Aufgaben im Gesundheitswesen gesteigert werden. \\
Das Ziel der Bachelorarbeit ist, die Entwicklung und Realisierung eines mehrschichtigen Softwaressystems,
zum erfassen von Messdaten und unterst\"utzung von Betreuer und Patient.




\section{Aufgabenstellung im Detail}

\subsection*{Praxisprojekt}

In dem Praxisprojekt soll eine Android App zum Monitoring von t\"aglichen Kalorienzufur von Patienten entwickelt und implementiert werden.
Die App soll die Handycamera als Barcode Scanner nutzen um Lebensmitteleigenschaften aus dem Internet von einer Datenbank abzufragen.
\"Uber die Differenzw\"agung und die Lebensmitteleigenschaften soll die entsprechende Kalorienzahl der entnomenen Menge
in einem Tagesprotokoll saldiert werden.
Die Wiegedaten sollen \"uber Bluetooth \"ubertragen werden. Die saldierten Tagesprotokolle
sollen zwecks sp\"aterer Auswertem\"oglichkeit in eine Webserver Datenbank \"ubertragen werden.
F\"ur die Auswertung soll eine Webapplikation entwickelt und implementiert werden welche die Tagesprotokolle \"ubersichtlich veranschaulicht.


\subsection*{Bachelorarbeit}

W\"ahrend der Bachelorarbeit soll die in dem Praxisprojekt erstellte Arbeit erweitert und verbessert werden.
Die Software soll Adipositas Erkrankten helfen, durch eine \"Ubersichtsfunktion den Di\"atplan einzuhalten.
Den behandelnden Arzt soll die Software erm\"oglichen das Ern\"ahrungsverhalten seiner Patienten zu \"uberwachen und bei einer ambulanten 
Adipositas F\"uhrung zu unterst\"utzen.

Die Tagesprotokolle sollen f\"ur eine aussagekr\"aftigere Diagnose erweitert werden. 
Zus\"atzlich zur Ermittlung der T\"aglichen Kalorienzufur soll 
nun auch das K\"orpergewicht und der Blutdruck inklusive Puls erfasst werden. 
F\"ur das Erfassen der Daten sollen eine Bluetooth-f\"ahige K\"orperwaage und ein Blutdruckmessger\"at genutzt werden. 
Um die User Experience zu verbessern soll die Webseitige Application unter Verwendung von Ajax-Architektur umgesetzt werden. 
Die Aufgezeichneten Daten sollen in einer Art Kalender, grafisch dargestellt werden. 
Der User in Arzt- Rolle soll die M\"oglichkeit haben die Aufl\"osung der dargestellten Daten zwischen Tag, Woche und Monat umschalten und 
\"uber ein Kalender- Widget mit Zusatztasten navigieren zu k\"onnen. 
Die Navigation soll Sprung zum gew\"unschten Datum und eine Verschiebung des Wochenanfangs und Wochenende erm\"oglichen.
Alle Daten sollen als Liniendiagramme dynamisch erstellt und visualisiert werden. Auch eine detalierte Einsicht in die Zusammens\"atzung der Mahlzeiten
soll entsprechend umgesetzt werden.
Als weiterer wichtiger Aspekt soll ein Konzept von Sicherheitsmechanismen f\"ur die Endpunkt zu Endpunkt Verschl\"usselung der Daten ausgearbeitet werden.
Das betrifft die Daten\"ubertragung vom Handy zum Server und vom Server zur Monitoring Software.
 

\section{Aufbau der Arbeit}

As part of the intership 

Kapitel1 \\

Kapitel2 \\

Kapitel3 \\
...